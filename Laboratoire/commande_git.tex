\documentclass{report}

\usepackage[utf8]{inputenc}
\usepackage[T1]{fontenc}
\usepackage[francais]{babel}
\title{Commandes git usuelles}
\author{Legrand Bruno}
\date{An de grâce 2019}

\begin{document}
\maketitle

\part{Workflow git}
\chapter{Préparation d'une nouvelle branche}
\section{Revenir à la branche master}
git checkout master

\section{faire retourner un fichier dans l'état du dernier commit}
git fetch upstream

Permet de récupérer l'information (mise à jour) de upstream mais on ne veut pas encore l'avoir dans notre branche

\section{Appliquer les modifications de upstream à sa branche}
git pull upstream master

\section{Appliquer les modifications de sa branche à  upstream}
git push upstream master

\section{ajouter un dépôt distant}
git remote add upstream <URL\_GITHUB>

On peut maintenant utiliser le mot-clé upstream en lieu et place de l'URL complète

\section{Afficher les dépôts distants}
git remote -v


\section{Créer \& Changer de branche}
git checkout -b \og NOM\_DE\_LA\_BRANCHE \fg


\chapter{Code finalisé}
\paragraph{Une fois que le code a été testé et est près nous pouvons le push sur origin}

\section{ajouter le code au prochain commit}
git add \og . /src monfichier \fg

\section{Envoyer le commit avec un message}
git commit -m "MON\_MESSAGE"

\section{push le commit de la branch sur le repo origin github}
git push origin <NOM\_DE\_LA\_BRANCHE>


\end{document}
